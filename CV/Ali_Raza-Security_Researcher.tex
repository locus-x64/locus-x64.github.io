\documentclass[a4paper,11pt]{article}

% Package imports
\usepackage{latexsym}
\usepackage{xcolor}
\usepackage{float}
\usepackage{ragged2e}
\usepackage[empty]{fullpage}
\usepackage{wrapfig}
\usepackage{lipsum}
\usepackage{tabularx}
\usepackage{titlesec}
\usepackage{geometry}
\usepackage{marvosym}
\usepackage{verbatim}
\usepackage{enumitem}
\usepackage{fancyhdr}
\usepackage{multicol}
\usepackage{graphicx}
\usepackage{cfr-lm}
\usepackage[T1]{fontenc}
\usepackage{fontawesome5}

% Color definitions
\definecolor{darkblue}{RGB}{0,0,139}

% Page layout
\setlength{\multicolsep}{0pt} 
\pagestyle{fancy}
\fancyhf{} % clear all header and footer fields
\fancyfoot{}
\renewcommand{\headrulewidth}{0pt}
\renewcommand{\footrulewidth}{0pt}
\geometry{left=1.4cm, top=0.8cm, right=1.2cm, bottom=1cm}
\setlength{\footskip}{5pt} % Addressing fancyhdr warning

% Hyperlink setup (moved after fancyhdr to address warning)
\usepackage[hidelinks]{hyperref}
\hypersetup{
    colorlinks=true,
    linkcolor=darkblue,
    filecolor=darkblue,
    urlcolor=darkblue,
}

% Custom box settings
\usepackage[most]{tcolorbox}
\tcbset{
    frame code={},
    center title,
    left=0pt,
    right=0pt,
    top=0pt,
    bottom=0pt,
    colback=gray!20,
    colframe=white,
    width=\dimexpr\textwidth\relax,
    enlarge left by=-2mm,
    boxsep=4pt,
    arc=0pt,outer arc=0pt,
}

% URL style
\urlstyle{same}

% Text alignment
\raggedright
\setlength{\tabcolsep}{0in}

% Section formatting
\titleformat{\section}{
  \vspace{-4pt}\scshape\raggedright\large
}{}{0em}{}[\color{black}\titlerule \vspace{-7pt}]

% Custom commands
\newcommand{\resumeItem}[2]{
  \item{
    \textbf{#1}{\hspace{0.5mm}#2 \vspace{-0.5mm}}
  }
}

\newcommand{\resumePOR}[3]{
\vspace{0.5mm}\item
    \begin{tabular*}{0.97\textwidth}[t]{l@{\extracolsep{\fill}}r}
        \textbf{#1}\hspace{0.3mm}#2 & \textit{\small{#3}} 
    \end{tabular*}
    \vspace{-2mm}
}

\newcommand{\resumeSubheading}[4]{
\vspace{0.5mm}\item
    \begin{tabular*}{0.98\textwidth}[t]{l@{\extracolsep{\fill}}r}
        \textbf{#1} & \textit{\footnotesize{#4}} \\
        \textit{\footnotesize{#3}} &  \footnotesize{#2}\\
    \end{tabular*}
    \vspace{-2.4mm}
}

\newcommand{\resumeProject}[4]{
\vspace{0.5mm}\item
    \begin{tabular*}{0.98\textwidth}[t]{l@{\extracolsep{\fill}}r}
        \textbf{#1} & \textit{\footnotesize{#3}} \\
        \footnotesize{\textit{#2}} & \footnotesize{#4}
    \end{tabular*}
    \vspace{-2.4mm}
}

\newcommand{\resumeResearchExp}[2]{
\vspace{0.5mm}\item
        \textbf{#1} 
        {#2}
    % \vspace{-2.4mm}
}

\newcommand{\resumeSubItem}[2]{\resumeItem{#2}{#2}\vspace{-4pt}}

\renewcommand{\labelitemi}{$\vcenter{\hbox{\tiny$\bullet$}}$}
\renewcommand{\labelitemii}{$\vcenter{\hbox{\tiny$\circ$}}$}

\newcommand{\resumeSubHeadingListStart}{\begin{itemize}[leftmargin=*,labelsep=1mm]}
\newcommand{\resumeHeadingSkillStart}{\begin{itemize}[leftmargin=*,itemsep=1.7mm, rightmargin=2ex]}
\newcommand{\resumeItemListStart}{\begin{itemize}[leftmargin=*,labelsep=1mm,itemsep=0.5mm]}

\newcommand{\resumeSubHeadingListEnd}{\end{itemize}\vspace{2mm}}
\newcommand{\resumeHeadingSkillEnd}{\end{itemize}\vspace{-2mm}}
\newcommand{\resumeItemListEnd}{\end{itemize}\vspace{-2mm}}
\newcommand{\cvsection}[1]{%
\vspace{2mm}
\begin{tcolorbox}
    \textbf{\large #1}
\end{tcolorbox}
    \vspace{-4mm}
}

\newcolumntype{L}{>{\raggedright\arraybackslash}X}%
\newcolumntype{R}{>{\raggedleft\arraybackslash}X}%
\newcolumntype{C}{>{\centering\arraybackslash}X}%

% Commands for icon sizing and positioning
\newcommand{\socialicon}[1]{\raisebox{-0.05em}{\resizebox{!}{1em}{#1}}}
\newcommand{\ieeeicon}[1]{\raisebox{-0.3em}{\resizebox{!}{1.3em}{#1}}}

% Font options
\newcommand{\headerfonti}{\fontfamily{phv}\selectfont} % Helvetica-like (similar to Arial/Calibri)
\newcommand{\headerfontii}{\fontfamily{ptm}\selectfont} % Times-like (similar to Times New Roman)
\newcommand{\headerfontiii}{\fontfamily{ppl}\selectfont} % Palatino (elegant serif)
\newcommand{\headerfontiv}{\fontfamily{pbk}\selectfont} % Bookman (readable serif)
\newcommand{\headerfontv}{\fontfamily{pag}\selectfont} % Avant Garde-like (similar to Trebuchet MS)
\newcommand{\headerfontvi}{\fontfamily{cmss}\selectfont} % Computer Modern Sans Serif
\newcommand{\headerfontvii}{\fontfamily{qhv}\selectfont} % Quasi-Helvetica (another Arial/Calibri alternative)
\newcommand{\headerfontviii}{\fontfamily{qpl}\selectfont} % Quasi-Palatino (another elegant serif option)
\newcommand{\headerfontix}{\fontfamily{qtm}\selectfont} % Quasi-Times (another Times New Roman alternative)
\newcommand{\headerfontx}{\fontfamily{bch}\selectfont} % Charter (clean serif font)

\begin{document}
\headerfontiii

% Header
\begin{center}
    {\Huge\textbf{ALI RAZA}}\\
    {\textit{Security Researcher}}
\end{center}
\vspace{-6mm}

\begin{center}
    \small{
    +92-305-738-1431 | \href{mailto:elirazamumtaz@gmail.com}{elirazamumtaz@gmail.com} | 
    \href{https://locus-x64.github.io/}{locus-x64.github.io}
    }
\end{center}
\vspace{-6mm}

\begin{center}
    \small{
    \socialicon{\faLinkedin} \href{https://www.linkedin.com/in/alirazamumtaz/}{alirazamumtaz} | 
    \socialicon{\faGithub} \href{https://github.com/locus-x64}{locus-x64} |
    \socialicon{\faTwitter} \href{https://twitter.com/locus_x64}{locus\_x64}
    }
\end{center}
\vspace{-6mm}
\begin{center}
    \small{Lahore, Punjab - 53000, Pakistan}
\end{center}

\vspace{-4mm}

% \section{\textbf{Objective}}
% \vspace{1mm}
% \small{
% I am interested in identifying security vulnerabilities posed by operating systems and understanding their potential exploitation. At Ebryx, I have worked on Linux kernel exploitation with focus on n-day research. Now we are building kernel-level security measures to block exploit techniques.
% }
% \vspace{-2mm}



\section{\textbf{Professional Experience}}
\vspace{-0.4mm}
  \resumeSubHeadingListStart
   % \resumeSubHeadingListStart
  \resumeSubheading
      {{Ebryx (Pvt.) Ltd. [\href{https://www.ebryx.com}{\faIcon{globe}}]}}{Lahore, Pakistan}
      {Security Researcher}{Mar 2023 - Current}
      \resumeItemListStart
        \item Mitigating attacks by performing interpreter \& runtime hardening
        \item Designed kernel-level technique using Linux netfilters to detect path traversal attacks on a Linux system
        \item Designed userland agent using JVMTI to detect Java deserialization attacks on a Linux system
        \item Designed kernel-level technique using LKMs to detect ASLR brute force attacks on a Linux system
        \item Discovered a 0-day vulnerability (CVE-2024-22857) in the open-source logging library zlog using AFL++ fuzzing
        \item Performed fuzzing on Linux kernel-specific syscalls using syzkaller, focusing on black-box security research
        \item Conducted n-day research on Linux Kernel Exploitation, improving security assessments and attack strategies
        \item Formalized a Linux kernel exploitation attack matrix, uncovering exploitable kernel objects and refining pre/post-exploitation techniques
      \resumeItemListEnd
 % \resumeSubHeadingListEnd
  % \resumeSubheading
  %     {{Ebryx (Pvt.) Ltd. [\href{https://www.ebryx.com}{\faIcon{globe}}]}}{Lahore, Pakistan}
  %     {Security Researcher}{Mar 2023 - Current}
  %     \resumeItemListStart
  %       \item Working on Full Stack Attack Chains on Linux Environment building security mechanisms on system level
  %       \item Found 0-day vulnerability as CVE-2024-22857 in a famous open-source logging library
  %       \item Worked on syzkaller to fuzz Linux kernel's specific syscalls
  %       \item Worked on Linux Kernel Exploitation (n-Day Research)
  %     \resumeItemListEnd 
  \resumeSubheading
    {University of the Punjab [\href{https://www.pu.edu.pk}{\faIcon{globe}}]}{Lahore, Pakistan}
    {Teaching Assistant}{Oct 2022 - Feb 2023}
    \resumeItemListStart
      \item Designed material and coursework for the newly introduced lab component of the subject
      \item Designed exam papers for the lab
      \item Assisted students in the lab + other TA responsibilities
    \resumeItemListEnd
  \resumeSubHeadingListEnd
\vspace{-6mm}

\section{\textbf{Research Experience}}
\vspace{-0.4mm}
\resumeSubHeadingListStart

\resumeResearchExp
  {n-day ("Call of Death" in Shannon Baseband) - CVE-2020-25279 [\href{https://cve.mitre.org/cgi-bin/cvename.cgi?name=CVE-2020-25279}{\faIcon{globe}}] }
{
\resumeItemListStart
  \item Looked into Samsung's Exynos modem chip that uses Shannon RTOS
  \item Used IDA Python and Ghidra scripts combined to load the firmware file for reversing
  \item Analysed the PAL memory allocation mechanism in Shannon
  \item Found the vulnerable code for the CVE mentioned above statically
  \item Used FirmWire to emulate the firmware
  \item Tools used: FirmWire, IDA Pro 9-beta, Ghidra
\resumeItemListEnd
}

\resumeResearchExp
  {0-day in Zlog: CVE-2024-22857 [\href{https://www.cybersecurity-help.cz/vdb/SB2024022842}{\faIcon{globe}}] }
{
\resumeItemListStart
  \item Conducted fuzzing of zlog, leading to the discovery of a critical 0-day vulnerability (CVE-2024-22857)
  \item Successfully identified and reported the vulnerability, which allowed arbitrary code execution
  \item Developed proof-of-concept (PoC) exploit to demonstrate the feasibility of the attack and assisted in proposing mitigations
  \item Collaborated with the vendor to ensure a timely patch and public disclosure of the vulnerability
  \item Tools used: AFL++, elixir, gdb, git
\resumeItemListEnd
}

\resumeResearchExp
  {n-day (Dirty Pipe) - CVE-2022-0847 [\href{https://cve.mitre.org/cgi-bin/cvename.cgi?name=CVE-2020-25279}{\faIcon{globe}}] }
{
\resumeItemListStart
  \item Explored different data-only attacks in Linux kernel
  \item Looked into the in-memory buffer management inside kernel
  \item Following the source of pipe IPC in Linux kernel using elixir.bootlin, wrote a PoC for the CVE-2022-0847
  \item Tools used: Elixir Bootlin, GDB with bata24/gef, QEMU
\resumeItemListEnd
}

\resumeResearchExp
  {Vulnerability Research \& Exploit Development for Android Kernel [\href{https://github.com/locus-x64/vred-notes}{\faIcon{globe}}] }
{
\resumeItemListStart
  \item Final Year Project (FYP) during Bachelor
  \item Supervised by Dr. Muhammad Arif Butt \href{https://arif.phd}{(arif.phd)}
  \item Started binary exploitation from Linux user-land and completed with kernel-land exploitation
  \item Conducted n-day research on CVE-2019-2215
\resumeItemListEnd
}
\vspace{-7mm}
\resumeSubHeadingListEnd

\section{\textbf{Skills}}
\vspace{-0.4mm}
 \resumeHeadingSkillStart
  \resumeItem{Programming: }
    {ANSI C, Assembly x86-64/ARM, Bash, Python}
  \resumeItem{Research: }
    {Linux Kernel, Mobile Baseband, Android Kernel, Linux Runtime, Python Interpreter, JVMTi}
  \resumeItem{Tools: }
    {QEMU, VMWare Workstation, IDA Pro \href{https://p.ost2.fyi/certificates/dac47c9287304e249e34062bd568ffe7}{(ost2 certified)}, Ghidra, GDB with gef, AFL++, elixir, CodeQL, Kali Toolchain, FlareVM Toolchain }
  \resumeItem{Operating System: }
    {Linux (Ubuntu), Android}
    \resumeItem{Open Source Contributions: }
    {zlog(vulnerability patch), Elixir Core Reference, Havoc (C2) Framework,  pwncollege, Hacktoberfest contributor}

 \resumeHeadingSkillEnd

\section{\textbf{Education}}
\vspace{-0.4mm}
\resumeSubHeadingListStart

\resumeSubheading
{PUCIT, University of the Punjab}{Lahore, Pakistan}
{Bachelor of Computer Science}{Oct 2019 - July 2023}
\resumeItemListStart
\item GPA: 3.58/4.00
\item Campus Lead by Google Developer Student Clubs [\href{http://g.dev/alirazamumtaz}{\faIcon{globe}}]
\item President of PUCon23 (National Tech Event by University of the Punjab) [\href{https://pucit.edu.pk/national-programming-competition-pucon23/}{\faIcon{globe}}]
\resumeItemListEnd

\resumeSubheading
{Punjab Group of Colleges}{Okara, Pakistan}
{Intermediate of Computer Science (ICS)}{Aug 2017 - Oct 2019}
\resumeItemListStart
\item Grade: 90.54\%
\item Board Topper [\href{https://pgc.edu/wp-content/uploads/2019/10/69807212_2482536158459889_3093261925052579840_o_2482536155126556.png}{\faIcon{globe}}]
\resumeItemListEnd

\resumeSubHeadingListEnd
\vspace{-6mm}



\section{\textbf{University Projects}}
\vspace{-0.4mm}
\resumeSubHeadingListStart

\resumeProject
  {Unix Shell]}
  {Tools: C, gdb, Makefile, Linux Syscalls}
  {}
  {{}[\href{https://github.com/locus-x64/system-programming/tree/master/linux-shell}{\textcolor{darkblue}{\faGithub}}]}
\resumeItemListStart
  \item An effort to write the *nix-based shell to gain an understanding of how the shell works and how OS creates and handles processes and allows processes to communicate with each other through its IPC interface
\resumeItemListEnd

\resumeProject
  {Exploit Scripts}
  {Tools: C, Python, x86-64 Assembly}
  {}
  {{}[\href{https://github.com/locus-x64/exploit-development}{\textcolor{darkblue}{\faGithub}}]}
\resumeItemListStart
  \item Basic scripts that I have written to solve some exploitation challenges
\resumeItemListEnd

\resumeProject
  {Hack Assembler}
  {Tools: C++, gdb}
  {}
  {{}[\href{https://github.com/locus-x64/hack-assembler}{\textcolor{darkblue}{\faGithub}}]}
\resumeItemListStart
  \item A 16-bit machine language assembler for the 16-bit Hack Assembly Language. It was done as part of building a complete 16-bit computer during the Computer Organization Assembly Language Course
\resumeItemListEnd

\resumeSubHeadingListEnd



% \section{\textbf{Honors and Awards}}
% \vspace{-0.4mm}
% \resumeSubHeadingListStart

% \resumeProject
%   {Award Name A}
%   {Awarding Institution/Organization}
%   {Month Year}
%   {{}[\href{https://award-link-a.com}{\textcolor{darkblue}{\faIcon{globe}}}]}
% \resumeItemListStart
%   \item Brief description of the award and its significance
%   \item Impact or recognition associated with the award
% \resumeItemListEnd

% \resumeProject
%   {Award Name B}
%   {Awarding Institution/Organization}
%   {Month Year}
%   {{}[\href{https://award-link-b.com}{\textcolor{darkblue}{\faIcon{globe}}}]}
% \resumeItemListStart
%   \item Brief description of the award and its significance
%   \item Impact or recognition associated with the award
% \resumeItemListEnd

% \resumeProject
%   {Competition Achievement}
%   {Competition Name, Organizing Body}
%   {Month Year}
%   {{}[\href{https://competition-link.com}{\textcolor{darkblue}{\faIcon{globe}}}]}
% \resumeItemListStart
%   \item Specific achievement or rank in the competition
%   \item Skills or abilities demonstrated through this achievement
% \resumeItemListEnd

% \resumeSubHeadingListEnd

% \vspace{-6mm}
% \section{\textbf{Leadership Experience}}
% \vspace{-0.4mm}
% \resumeSubHeadingListStart
% \resumeProject
%   {Leadership Role A}
%   {Organization/Institution Name}
%   {Month Year - Month Year}
%   {{}[\href{https://organization-a-link.com}{\textcolor{darkblue}{\faIcon{globe}}}]}
% \resumeItemListStart
%   \item Key responsibility or achievement in this role
%   \item Quantifiable impact or improvement made during tenure
%   \item Initiative taken or project led
% \resumeItemListEnd

% \resumeProject
%   {Leadership Role B}
%   {Organization/Institution Name}
%   {Month Year - Month Year}
%   {{}[\href{https://organization-b-link.com}{\textcolor{darkblue}{\faIcon{globe}}}]}
% \resumeItemListStart
%   \item Key responsibility or achievement in this role
%   \item Quantifiable impact or improvement made during tenure
%   \item Initiative taken or project led
% \resumeItemListEnd

% \resumeSubHeadingListEnd

% \vspace{-6mm}

% \section{\textbf{Volunteer Experience}}
% \vspace{-0.4mm}
% \resumeSubHeadingListStart
% \resumeProject
%   {Volunteer Role A}
%   {Organization Name}
%   {Month Year - Month Year}
%   {{}[\href{https://volunteer-org-a-link.com}{\textcolor{darkblue}{\faIcon{globe}}}]}
% \resumeItemListStart
%   \item Key responsibility or contribution in this role
%   \item Impact of your volunteer work
%   \item Skills developed or applied during this experience
% \resumeItemListEnd

% \resumeProject
%   {Volunteer Role B}
%   {Organization Name}
%   {Month Year - Present}
%   {{}[\href{https://volunteer-org-b-link.com}{\textcolor{darkblue}{\faIcon{globe}}}]}
% \resumeItemListStart
%   \item Key responsibility or contribution in this role
%   \item Impact of your volunteer work
%   \item Skills developed or applied during this experience
% \resumeItemListEnd

% \resumeSubHeadingListEnd
% \vspace{-6mm}

% \section{\textbf{Professional Memberships}}
% \vspace{-0.4mm}
% \resumeSubHeadingListStart
% \resumePOR{Professional Organization A}
%     {, Membership ID: XXXXXXXX}
%     {Month Year - Present}
% \resumePOR{Professional Organization B}
%     {, \href{https://membership-certificate-link.com}{Membership ID: XXXXXXXX}}
%     {Month Year - Present}
% \resumePOR{Professional Organization C}
%     {, \href{https://membership-certificate-link.com}{Membership ID: XXXXXXXX}}
%     {Month Year - Present}

% \resumeSubHeadingListEnd
% \vspace{-6mm}

% \section{\textbf{Certifications}}
% \vspace{-0.2mm}
% \resumeSubHeadingListStart
% \resumePOR{}{\href{https://certification-link-a.com}{
% \textbf{Certification A}
% }}{Month Year}
% \resumePOR{}{
% \textbf{Certifying Body:} {{\href{https://certification-link-b.com}{Certification B}}}}{Month Year}
% \resumePOR{}{
% \textbf{Certifying Body:} {{\href{https://certification-link-c.com}{Certification C}}}}{Month Year}
% \resumePOR{}{\href{https://certification-link-d.com}{
% \textbf{Certification D}
% }}{Month Year}

% \resumeSubHeadingListEnd
% \vspace{-6mm}

\end{document}