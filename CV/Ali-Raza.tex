\documentclass[a4paper,11pt]{article}

% Package imports
\usepackage{latexsym}
\usepackage{xcolor}
\usepackage{float}
\usepackage{ragged2e}
\usepackage[empty]{fullpage}
\usepackage{wrapfig}
\usepackage{lipsum}
\usepackage{tabularx}
\usepackage{titlesec}
\usepackage{geometry}
\usepackage{marvosym}
\usepackage{verbatim}
\usepackage{enumitem}
\usepackage{fancyhdr}
\usepackage{multicol}
\usepackage{graphicx}
\usepackage{cfr-lm}
\usepackage[T1]{fontenc}
\usepackage{fontawesome5}

% Color definitions
\definecolor{darkblue}{RGB}{0,0,139}

% Page layout
\setlength{\multicolsep}{0pt} 
\pagestyle{fancy}
\fancyhf{} % clear all header and footer fields
\fancyfoot{}
\renewcommand{\headrulewidth}{0pt}
\renewcommand{\footrulewidth}{0pt}
\geometry{left=1.4cm, top=0.8cm, right=1.2cm, bottom=1cm}
\setlength{\footskip}{5pt} % Addressing fancyhdr warning

% Hyperlink setup (moved after fancyhdr to address warning)
\usepackage[hidelinks]{hyperref}
\hypersetup{
    colorlinks=true,
    linkcolor=darkblue,
    filecolor=darkblue,
    urlcolor=darkblue,
}

% Custom box settings
\usepackage[most]{tcolorbox}
\tcbset{
    frame code={},
    center title,
    left=0pt,
    right=0pt,
    top=0pt,
    bottom=0pt,
    colback=gray!20,
    colframe=white,
    width=\dimexpr\textwidth\relax,
    enlarge left by=-2mm,
    boxsep=4pt,
    arc=0pt,outer arc=0pt,
}

% URL style
\urlstyle{same}

% Text alignment
\raggedright
\setlength{\tabcolsep}{0in}

% Section formatting
\titleformat{\section}{
  \vspace{-4pt}\scshape\raggedright\large
}{}{0em}{}[\color{black}\titlerule \vspace{-7pt}]

% Custom commands
\newcommand{\resumeItem}[2]{
  \item{
    \textbf{#1}{\hspace{0.5mm}#2 \vspace{-0.5mm}}
  }
}

\newcommand{\resumePOR}[3]{
\vspace{0.5mm}\item
    \begin{tabular*}{0.97\textwidth}[t]{l@{\extracolsep{\fill}}r}
        \textbf{#1}\hspace{0.3mm}#2 & \textit{\small{#3}} 
    \end{tabular*}
    \vspace{-2mm}
}

\newcommand{\resumeSubheading}[4]{
\vspace{0.5mm}\item
    \begin{tabular*}{0.98\textwidth}[t]{l@{\extracolsep{\fill}}r}
        \textbf{#1} & \textit{\footnotesize{#4}} \\
        \textit{\footnotesize{#3}} &  \footnotesize{#2}\\
    \end{tabular*}
    \vspace{-2.4mm}
}

\newcommand{\resumeProject}[4]{
\vspace{0.5mm}\item
    \begin{tabular*}{0.98\textwidth}[t]{l@{\extracolsep{\fill}}r}
        \textbf{#1} & \textit{\footnotesize{#3}} \\
        \footnotesize{\textit{#2}} & \footnotesize{#4}
    \end{tabular*}
    \vspace{-2.4mm}
}

\newcommand{\resumeResearchExp}[2]{
\vspace{0.5mm}\item
        \textbf{#1} 
        {#2}
    % \vspace{-2.4mm}
}

\newcommand{\resumeSubItem}[2]{\resumeItem{#2}{#2}\vspace{-4pt}}

\renewcommand{\labelitemi}{$\vcenter{\hbox{\tiny$\bullet$}}$}
\renewcommand{\labelitemii}{$\vcenter{\hbox{\tiny$\circ$}}$}

\newcommand{\resumeSubHeadingListStart}{\begin{itemize}[leftmargin=*,labelsep=1mm]}
\newcommand{\resumeHeadingSkillStart}{\begin{itemize}[leftmargin=*,itemsep=1.7mm, rightmargin=2ex]}
\newcommand{\resumeItemListStart}{\begin{itemize}[leftmargin=*,labelsep=1mm,itemsep=0.5mm]}

\newcommand{\resumeSubHeadingListEnd}{\end{itemize}\vspace{2mm}}
\newcommand{\resumeHeadingSkillEnd}{\end{itemize}\vspace{-2mm}}
\newcommand{\resumeItemListEnd}{\end{itemize}\vspace{-2mm}}
\newcommand{\cvsection}[1]{%
\vspace{2mm}
\begin{tcolorbox}
    \textbf{\large #1}
\end{tcolorbox}
    \vspace{-4mm}
}

\newcolumntype{L}{>{\raggedright\arraybackslash}X}%
\newcolumntype{R}{>{\raggedleft\arraybackslash}X}%
\newcolumntype{C}{>{\centering\arraybackslash}X}%

% Commands for icon sizing and positioning
\newcommand{\socialicon}[1]{\raisebox{-0.05em}{\resizebox{!}{1em}{#1}}}
\newcommand{\ieeeicon}[1]{\raisebox{-0.3em}{\resizebox{!}{1.3em}{#1}}}

% Font options
\newcommand{\headerfonti}{\fontfamily{phv}\selectfont} % Helvetica-like (similar to Arial/Calibri)
\newcommand{\headerfontii}{\fontfamily{ptm}\selectfont} % Times-like (similar to Times New Roman)
\newcommand{\headerfontiii}{\fontfamily{ppl}\selectfont} % Palatino (elegant serif)
\newcommand{\headerfontiv}{\fontfamily{pbk}\selectfont} % Bookman (readable serif)
\newcommand{\headerfontv}{\fontfamily{pag}\selectfont} % Avant Garde-like (similar to Trebuchet MS)
\newcommand{\headerfontvi}{\fontfamily{cmss}\selectfont} % Computer Modern Sans Serif
\newcommand{\headerfontvii}{\fontfamily{qhv}\selectfont} % Quasi-Helvetica (another Arial/Calibri alternative)
\newcommand{\headerfontviii}{\fontfamily{qpl}\selectfont} % Quasi-Palatino (another elegant serif option)
\newcommand{\headerfontix}{\fontfamily{qtm}\selectfont} % Quasi-Times (another Times New Roman alternative)
\newcommand{\headerfontx}{\fontfamily{bch}\selectfont} % Charter (clean serif font)

\begin{document}
\headerfontiii

% Header
\begin{center}
    {\Huge\textbf{ALI RAZA}}\\
    {\textit{Vulnerability Researcher}}
\end{center}
\vspace{-5mm}

\begin{center}
    \small{
    \href{mailto:elirazamumtaz@gmail.com}{elirazamumtaz@gmail.com} | 
    \href{https://locus-x64.github.io/}{locus-x64.github.io}
    }
\end{center}
\vspace{-5mm}

\begin{center}
    \small{
    \socialicon{\faLinkedin} \href{https://www.linkedin.com/in/locus-x64/}{locus-x64} | 
    \socialicon{\faGithub} \href{https://github.com/locus-x64}{locus-x64} |
    \socialicon{\faTwitter} \href{https://twitter.com/locus_x64}{locus\_x64}
    }
\end{center}
% \vspace{-6mm}
% \begin{center}
%     \small{Lahore, Punjab - 53000, Pakistan}
% \end{center}

\vspace{-4mm}

\section{\textbf{Objective}}
\vspace{1mm}
\small{
Security researcher with a strong background in C and assembly, focusing on fuzzing, reverse engineering, and code auditing to uncover and remediate software flaws. I develop robust PoCs, collaborate closely with cyber threat experts, and design practical mitigations across userland and kernel to enhance system security.
}
\vspace{-2mm}

\section{\textbf{Professional Experience}}
\vspace{-0.4mm}
  \resumeSubHeadingListStart
  \resumeSubheading
      {{Ebryx (Pvt.) Ltd. [\href{https://www.ebryx.com}{\faIcon{globe}}]}}{Lahore, Pakistan}
      {Vulnerability Researcher}{Mar 2023 - Current}
      \resumeItemListStart
        \item Collaborated with senior threat researchers to investigate vulnerabilities end-to-end and translate findings into actionable detections and mitigations
        \item Conducted targeted fuzzing (AFL++, syzkaller) across userland and Linux kernel; triaged crashes, minimized inputs, and authored PoCs
        \item Discovered and disclosed a 0-day in \texttt{python-socketio} (CVE-2025-61765) \href{https://github.com/miguelgrinberg/python-socketio/security/advisories/GHSA-g8c6-8fjj-2r4m}{[\faIcon{globe}]} with PoC and remediation guidance, coordinating with the maintainer
        \item Discovered and disclosed CVE-2024-22857 in zlog via AFL++; developed PoC exploit and proposed remediation, working with maintainers through coordinated disclosure
        \item Performed secure code reviews and static analysis of C/C++ codebases using CodeQL and manual auditing; hardened CPython against memory corruption classes
        \item Performed reverse engineering of firmware and system components with IDA Pro and Ghidra to pinpoint vulnerable code paths and exploitation primitives
        \item Designed kernel-level techniques (Netfilter, LKMs) to detect and mitigate path traversal and ASLR brute-force attacks on Linux
        \item Built a JVMTI-based userland agent to detect Java deserialization attack primitives at runtime on Linux
        \item Conducted n-day research in Linux kernel exploitation, and formalized an attack matrix mapping exploitable kernel objects, prerequisites, and post-exploitation techniques
      \resumeItemListEnd

  \resumeSubheading
    {University of the Punjab [\href{https://www.pu.edu.pk}{\faIcon{globe}}]}{Lahore, Pakistan}
    {Teaching Assistant}{Oct 2022 - Feb 2023}
    \resumeItemListStart
      \item Designed lab coursework and assessments; provided hands-on guidance and mentorship to students
    \resumeItemListEnd
  \resumeSubHeadingListEnd
\vspace{-6mm}

\section{\textbf{Research Experience}}
\vspace{-0.4mm}
\resumeSubHeadingListStart

\resumeResearchExp
  {0-day in \texttt{python-socketio}: CVE-2025-61765 \href{https://github.com/miguelgrinberg/python-socketio/security/advisories/GHSA-g8c6-8fjj-2r4m}{[\faIcon{globe}]}}
{
\resumeItemListStart
  \item Identified and reported a security flaw in \texttt{python-socketio}; reproduced impact with a PoC and supported mitigation guidance
  \item Collaborated with the maintainer for coordinated disclosure and release of a fix/advisory
  \item Tools used: Python, pytest, git
\resumeItemListEnd
}

\resumeResearchExp
  {0-day in Zlog: CVE-2024-22857 \href{https://www.cybersecurity-help.cz/vdb/SB2024022842}{[\faIcon{globe}]}}
{
\resumeItemListStart
  \item Fuzzed zlog and discovered a critical vulnerability enabling arbitrary code execution
  \item Built a PoC to demonstrate exploitability and collaborated on mitigation guidance
  \item Coordinated disclosure with the maintainer to patch and publish advisories
  \item Tools: AFL++, Elixir Bootlin, gdb, git
\resumeItemListEnd
}

\resumeResearchExp
  {n-day (Dirty Pipe) - CVE-2022-0847 \href{https://cve.mitre.org/cgi-bin/cvename.cgi?name=CVE-2022-0847}{[\faIcon{globe}]}}
{
\resumeItemListStart
  \item Explored data-only attacks and kernel buffer management internals
  \item Traced Linux pipe IPC via Elixir Bootlin and authored a working PoC
  \item Tools: Elixir Bootlin, GDB with bata24/gef, QEMU
\resumeItemListEnd
}

\resumeResearchExp
  {n-day ("Call of Death" in Shannon Baseband) - CVE-2020-25279 \href{https://cve.mitre.org/cgi-bin/cvename.cgi?name=CVE-2020-25279}{[\faIcon{globe}]}}
{
\resumeItemListStart
  \item Reversed Samsung Exynos modem firmware (Shannon RTOS) with IDA Python and Ghidra
  \item Analyzed PAL allocator and identified vulnerable code paths for the CVE statically
  \item Emulated the firmware with FirmWire to validate understanding and hypotheses
  \item Tools: FirmWire, IDA Pro 9-beta, Ghidra
\resumeItemListEnd
}

\resumeResearchExp
  {Vulnerability Research \& Exploit Development for Android Kernel \href{https://github.com/locus-x64/vred-notes}{[\faIcon{globe}]}}
{
\resumeItemListStart
  \item Final Year Project (FYP) supervised by Dr.\ Muhammad Arif Butt \href{https://arifbutt.me}{(arifbutt.me)}
  \item Progressed from Linux userland exploitation to Android/Linux kernel exploitation
  \item Conducted n-day research on CVE-2019-2215
\resumeItemListEnd
}
\vspace{-7mm}
\resumeSubHeadingListEnd

\section{\textbf{Skills}}
\vspace{-0.4mm}
 \resumeHeadingSkillStart
  \resumeItem{Programming: }
    {C (ANSI), Assembly (x86-64/ARM), Bash, Python}
  \resumeItem{Security Focus: }
    {Fuzzing, Reverse Engineering, Code Auditing (manual/CodeQL), Exploit Development, Mitigations}
  \resumeItem{Domains: }
    {Linux Kernel Internals, Android Kernel/Internals, Mobile Baseband, Python \& Java Runtimes (JVMTI)}
  \resumeItem{Tools: }
    {QEMU, VMware Workstation, IDA Pro \href{https://p.ost2.fyi/certificates/dac47c9287304e249e34062bd568ffe7}{(ost2 certified)}, Ghidra, GDB+gef, AFL++, Elixir Bootlin, CodeQL, Semgrep, Kali Toolchain, FlareVM Toolchain}
  \resumeItem{Operating Systems: }
    {Linux (Ubuntu), Android}
  \resumeItem{Open Source Contributions: }
    {zlog (CVE-2024-22857 patch), Elixir Core Reference, Havoc (C2) Framework, pwncollege, Hacktoberfest}
 \resumeHeadingSkillEnd

\section{\textbf{Education}}
\vspace{-0.4mm}
\resumeSubHeadingListStart

\resumeSubheading
{PUCIT, University of the Punjab}{Lahore, Pakistan}
{Bachelor of Computer Science}{Oct 2019 - July 2023}
\resumeItemListStart
\item Projects:
\begin{itemize}
    \item Vulnerability Research \& Exploit Development for Android Kernel [\href{https://github.com/locus-x64/vred-notes}{\faIcon{globe}}]
    \item UNIX Shell in C [\href{https://github.com/locus-x64/system-programming/tree/master/linux-shell}{\faIcon{globe}}]
    \item Hack Assembler in C++ [\href{https://github.com/locus-x64/hack-assembler}{\faIcon{globe}}]
    \item Exploit Scripts in C/Python [\href{https://github.com/locus-x64/exploit-development}{\faIcon{globe}}]
\end{itemize}
\item GPA: 3.58/4.00
\item Campus Lead by Google Developer Student Clubs [\href{https://g.dev/locus-x64}{\faIcon{globe}}]
\item President of PUCon23 (National Tech Event by University of the Punjab) [\href{https://pucit.edu.pk/national-programming-competition-pucon23/}{\faIcon{globe}}]
\resumeItemListEnd

\resumeSubheading
{Punjab Group of Colleges}{Okara, Pakistan}
{Intermediate of Computer Science (ICS)}{Aug 2017 - Oct 2019}
\resumeItemListStart
\item Grade: 90.54\%
\item Board Topper [\href{https://pgc.edu/wp-content/uploads/2019/10/69807212_2482536158459889_3093261925052579840_o_2482536155126556.png}{\faIcon{globe}}]
\resumeItemListEnd

\resumeSubHeadingListEnd
\vspace{-6mm}

\end{document}